% Options for packages loaded elsewhere
\PassOptionsToPackage{unicode}{hyperref}
\PassOptionsToPackage{hyphens}{url}
%
\documentclass[
  11pt,
]{article}
\usepackage{amsmath,amssymb}
\usepackage{iftex}
\ifPDFTeX
  \usepackage[T1]{fontenc}
  \usepackage[utf8]{inputenc}
  \usepackage{textcomp} % provide euro and other symbols
\else % if luatex or xetex
  \usepackage{unicode-math} % this also loads fontspec
  \defaultfontfeatures{Scale=MatchLowercase}
  \defaultfontfeatures[\rmfamily]{Ligatures=TeX,Scale=1}
\fi
\usepackage{lmodern}
\ifPDFTeX\else
  % xetex/luatex font selection
\fi
% Use upquote if available, for straight quotes in verbatim environments
\IfFileExists{upquote.sty}{\usepackage{upquote}}{}
\IfFileExists{microtype.sty}{% use microtype if available
  \usepackage[]{microtype}
  \UseMicrotypeSet[protrusion]{basicmath} % disable protrusion for tt fonts
}{}
\makeatletter
\@ifundefined{KOMAClassName}{% if non-KOMA class
  \IfFileExists{parskip.sty}{%
    \usepackage{parskip}
  }{% else
    \setlength{\parindent}{0pt}
    \setlength{\parskip}{6pt plus 2pt minus 1pt}}
}{% if KOMA class
  \KOMAoptions{parskip=half}}
\makeatother
\usepackage{xcolor}
\usepackage[margin=1in]{geometry}
\usepackage{color}
\usepackage{fancyvrb}
\newcommand{\VerbBar}{|}
\newcommand{\VERB}{\Verb[commandchars=\\\{\}]}
\DefineVerbatimEnvironment{Highlighting}{Verbatim}{commandchars=\\\{\}}
% Add ',fontsize=\small' for more characters per line
\usepackage{framed}
\definecolor{shadecolor}{RGB}{248,248,248}
\newenvironment{Shaded}{\begin{snugshade}}{\end{snugshade}}
\newcommand{\AlertTok}[1]{\textcolor[rgb]{0.94,0.16,0.16}{#1}}
\newcommand{\AnnotationTok}[1]{\textcolor[rgb]{0.56,0.35,0.01}{\textbf{\textit{#1}}}}
\newcommand{\AttributeTok}[1]{\textcolor[rgb]{0.13,0.29,0.53}{#1}}
\newcommand{\BaseNTok}[1]{\textcolor[rgb]{0.00,0.00,0.81}{#1}}
\newcommand{\BuiltInTok}[1]{#1}
\newcommand{\CharTok}[1]{\textcolor[rgb]{0.31,0.60,0.02}{#1}}
\newcommand{\CommentTok}[1]{\textcolor[rgb]{0.56,0.35,0.01}{\textit{#1}}}
\newcommand{\CommentVarTok}[1]{\textcolor[rgb]{0.56,0.35,0.01}{\textbf{\textit{#1}}}}
\newcommand{\ConstantTok}[1]{\textcolor[rgb]{0.56,0.35,0.01}{#1}}
\newcommand{\ControlFlowTok}[1]{\textcolor[rgb]{0.13,0.29,0.53}{\textbf{#1}}}
\newcommand{\DataTypeTok}[1]{\textcolor[rgb]{0.13,0.29,0.53}{#1}}
\newcommand{\DecValTok}[1]{\textcolor[rgb]{0.00,0.00,0.81}{#1}}
\newcommand{\DocumentationTok}[1]{\textcolor[rgb]{0.56,0.35,0.01}{\textbf{\textit{#1}}}}
\newcommand{\ErrorTok}[1]{\textcolor[rgb]{0.64,0.00,0.00}{\textbf{#1}}}
\newcommand{\ExtensionTok}[1]{#1}
\newcommand{\FloatTok}[1]{\textcolor[rgb]{0.00,0.00,0.81}{#1}}
\newcommand{\FunctionTok}[1]{\textcolor[rgb]{0.13,0.29,0.53}{\textbf{#1}}}
\newcommand{\ImportTok}[1]{#1}
\newcommand{\InformationTok}[1]{\textcolor[rgb]{0.56,0.35,0.01}{\textbf{\textit{#1}}}}
\newcommand{\KeywordTok}[1]{\textcolor[rgb]{0.13,0.29,0.53}{\textbf{#1}}}
\newcommand{\NormalTok}[1]{#1}
\newcommand{\OperatorTok}[1]{\textcolor[rgb]{0.81,0.36,0.00}{\textbf{#1}}}
\newcommand{\OtherTok}[1]{\textcolor[rgb]{0.56,0.35,0.01}{#1}}
\newcommand{\PreprocessorTok}[1]{\textcolor[rgb]{0.56,0.35,0.01}{\textit{#1}}}
\newcommand{\RegionMarkerTok}[1]{#1}
\newcommand{\SpecialCharTok}[1]{\textcolor[rgb]{0.81,0.36,0.00}{\textbf{#1}}}
\newcommand{\SpecialStringTok}[1]{\textcolor[rgb]{0.31,0.60,0.02}{#1}}
\newcommand{\StringTok}[1]{\textcolor[rgb]{0.31,0.60,0.02}{#1}}
\newcommand{\VariableTok}[1]{\textcolor[rgb]{0.00,0.00,0.00}{#1}}
\newcommand{\VerbatimStringTok}[1]{\textcolor[rgb]{0.31,0.60,0.02}{#1}}
\newcommand{\WarningTok}[1]{\textcolor[rgb]{0.56,0.35,0.01}{\textbf{\textit{#1}}}}
\usepackage{longtable,booktabs,array}
\usepackage{calc} % for calculating minipage widths
% Correct order of tables after \paragraph or \subparagraph
\usepackage{etoolbox}
\makeatletter
\patchcmd\longtable{\par}{\if@noskipsec\mbox{}\fi\par}{}{}
\makeatother
% Allow footnotes in longtable head/foot
\IfFileExists{footnotehyper.sty}{\usepackage{footnotehyper}}{\usepackage{footnote}}
\makesavenoteenv{longtable}
\usepackage{graphicx}
\makeatletter
\def\maxwidth{\ifdim\Gin@nat@width>\linewidth\linewidth\else\Gin@nat@width\fi}
\def\maxheight{\ifdim\Gin@nat@height>\textheight\textheight\else\Gin@nat@height\fi}
\makeatother
% Scale images if necessary, so that they will not overflow the page
% margins by default, and it is still possible to overwrite the defaults
% using explicit options in \includegraphics[width, height, ...]{}
\setkeys{Gin}{width=\maxwidth,height=\maxheight,keepaspectratio}
% Set default figure placement to htbp
\makeatletter
\def\fps@figure{htbp}
\makeatother
\setlength{\emergencystretch}{3em} % prevent overfull lines
\providecommand{\tightlist}{%
  \setlength{\itemsep}{0pt}\setlength{\parskip}{0pt}}
\setcounter{secnumdepth}{5}
\usepackage{float}
\usepackage{fancyhdr}
\pagestyle{fancy}
\fancyhead[L]{PathogenicityRanking Documentation}
\fancyhead[R]{\thepage}
\fancyfoot[C]{Mohammad Deen Hayatu - June 2025}
\ifLuaTeX
  \usepackage{selnolig}  % disable illegal ligatures
\fi
\usepackage{bookmark}
\IfFileExists{xurl.sty}{\usepackage{xurl}}{} % add URL line breaks if available
\urlstyle{same}
\hypersetup{
  pdftitle={PathogenicityRanking R Package Documentation},
  pdfauthor={Mohammad Deen Hayatu},
  hidelinks,
  pdfcreator={LaTeX via pandoc}}

\title{PathogenicityRanking R Package Documentation}
\usepackage{etoolbox}
\makeatletter
\providecommand{\subtitle}[1]{% add subtitle to \maketitle
  \apptocmd{\@title}{\par {\large #1 \par}}{}{}
}
\makeatother
\subtitle{Complete Setup and Implementation Guide}
\author{Mohammad Deen Hayatu}
\date{2025-06-26}

\begin{document}
\maketitle

{
\setcounter{tocdepth}{3}
\tableofcontents
}
\newpage

\section{Project Overview}\label{project-overview}

\subsection{Package Information}\label{package-information}

\textbf{Package Name}: PathogenicityRanking\\
\textbf{Purpose}: R package for computing composite pathogenicity scores
to rank missense variants\\
\textbf{Author}: Mohammad Deen Hayatu\\
\textbf{GitHub Repository}:
\url{https://github.com/MohammadDeen/PathogenicityRanking}\\
\textbf{License}: MIT\\
\textbf{Version}: 0.1.0\\
\textbf{Date Created}: June 26, 2025

\subsection{Abstract}\label{abstract}

The PathogenicityRanking package provides a comprehensive solution for
evaluating and ranking missense variants based on composite
pathogenicity scores derived from multiple predictors including
AlphaMissense, CADD, GERP++, phyloP, MPC, REVEL, and MetaSVM scores.
This package streamlines the variant analysis workflow by automating
data processing, score normalization, composite scoring, and
visualization generation.

\newpage

\section{Repository Structure}\label{repository-structure}

The PathogenicityRanking repository follows standard R package
conventions with additional GitHub integration features:

\begin{verbatim}
PathogenicityRanking/
├── .github/
│   └── workflows/
│       └── R-CMD-check.yaml          # GitHub Actions CI/CD workflow
├── R/
│   └── run_pathogenicity_analysis.R  # Main function implementation
├── vignettes/
│   └── PathogenicityRankingTutorial.Rmd  # Package tutorial
├── docs/                             # pkgdown documentation site
├── .gitignore                        # Git ignore rules for R packages
├── DESCRIPTION                       # Package metadata and dependencies
├── LICENSE                           # MIT license file
├── NAMESPACE                         # Package exports
├── README.md                         # Main documentation and usage
├── index.md                          # Package index
└── pkgdown.yml                       # pkgdown configuration
\end{verbatim}

\subsection{Key Files Description}\label{key-files-description}

\subsubsection{DESCRIPTION File}\label{description-file}

Contains package metadata, dependencies, and licensing information
following R package standards.

\subsubsection{NAMESPACE File}\label{namespace-file}

Exports the main function \texttt{run\_pathogenicity\_analysis()} for
public use.

\subsubsection{README.md}\label{readme.md}

Comprehensive documentation including installation instructions, usage
examples, and GitHub badges for CI status and licensing.

\newpage

\section{Package Components}\label{package-components}

\subsection{Main Function:
run\_pathogenicity\_analysis()}\label{main-function-run_pathogenicity_analysis}

\subsubsection{Function Signature}\label{function-signature}

\begin{Shaded}
\begin{Highlighting}[]
\FunctionTok{run\_pathogenicity\_analysis}\NormalTok{(input\_file, pdf\_output, png\_output, }\AttributeTok{show\_plot =} \ConstantTok{TRUE}\NormalTok{)}
\end{Highlighting}
\end{Shaded}

\subsubsection{Parameters}\label{parameters}

\begin{longtable}[]{@{}
  >{\raggedright\arraybackslash}p{(\columnwidth - 4\tabcolsep) * \real{0.3667}}
  >{\raggedright\arraybackslash}p{(\columnwidth - 4\tabcolsep) * \real{0.2000}}
  >{\raggedright\arraybackslash}p{(\columnwidth - 4\tabcolsep) * \real{0.4333}}@{}}
\toprule\noalign{}
\begin{minipage}[b]{\linewidth}\raggedright
Parameter
\end{minipage} & \begin{minipage}[b]{\linewidth}\raggedright
Type
\end{minipage} & \begin{minipage}[b]{\linewidth}\raggedright
Description
\end{minipage} \\
\midrule\noalign{}
\endhead
\bottomrule\noalign{}
\endlastfoot
\texttt{input\_file} & character & Path to input variant file (.csv,
.txt, .xlsx) \\
\texttt{pdf\_output} & character & Output filename for PDF plot \\
\texttt{png\_output} & character & Output filename for PNG plot \\
\texttt{show\_plot} & logical & Whether to display plot in RStudio
(default: TRUE) \\
\end{longtable}

\subsubsection{Required Input Columns}\label{required-input-columns}

The input file must contain the following columns with exact naming:

\begin{enumerate}
\def\labelenumi{\arabic{enumi}.}
\tightlist
\item
  \texttt{AAChange.refGeneWithVer} - Amino acid change annotation
\item
  \texttt{AlphaMissense\_score} - AlphaMissense pathogenicity score
\item
  \texttt{CADD\_phred} - CADD Phred-scaled score
\item
  \texttt{GERP++\_RS} - GERP++ rejection substitution score
\item
  \texttt{phyloP17way\_primate} - PhyloP conservation score
\item
  \texttt{MPC\_score} - Missense badness, PolyPhen-2, and Constraint
  score
\item
  \texttt{REVEL\_score} - Rare Exome Variant Ensemble Learner score
\item
  \texttt{MetaSVM\_score} - MetaSVM pathogenicity prediction score
\end{enumerate}

\subsubsection{Function Workflow}\label{function-workflow}

\begin{enumerate}
\def\labelenumi{\arabic{enumi}.}
\tightlist
\item
  \textbf{File Validation}: Checks file existence and format support
\item
  \textbf{Data Loading}: Reads CSV, TXT, or XLSX files using appropriate
  libraries
\item
  \textbf{Column Selection}: Extracts required columns and renames for
  processing
\item
  \textbf{Data Filtering}: Removes variants with missing AlphaMissense
  or CADD scores
\item
  \textbf{Score Normalization}: Normalizes each score to 0-1 range using
  max normalization
\item
  \textbf{Composite Scoring}: Calculates mean of normalized scores per
  variant
\item
  \textbf{Ranking}: Sorts variants by composite score in descending
  order
\item
  \textbf{Visualization}: Creates horizontal bar chart with professional
  styling
\item
  \textbf{Export}: Saves results as CSV and plots in PDF/PNG formats
\end{enumerate}

\subsubsection{Return Value}\label{return-value}

Returns a data frame containing: - Original variant identifiers -
Normalized individual scores - Composite pathogenicity scores - Variants
sorted by composite score (highest to lowest)

\newpage

\section{Dependencies}\label{dependencies}

\subsection{Required R Packages}\label{required-r-packages}

The PathogenicityRanking package depends on the following R packages:

\subsubsection{Core Dependencies}\label{core-dependencies}

\begin{itemize}
\tightlist
\item
  \textbf{dplyr} (\textgreater= 1.0.0): Data manipulation and
  transformation
\item
  \textbf{ggplot2} (\textgreater= 3.0.0): Data visualization and
  plotting
\item
  \textbf{readr} (\textgreater= 2.0.0): Fast and friendly CSV file
  reading
\item
  \textbf{readxl} (\textgreater= 1.3.0): Excel file reading capabilities
\item
  \textbf{fs} (\textgreater= 1.5.0): Cross-platform file system
  operations
\end{itemize}

\subsubsection{Dependency Installation}\label{dependency-installation}

These dependencies are automatically installed when the package is
installed via \texttt{devtools::install\_github()}.

\subsection{System Requirements}\label{system-requirements}

\begin{itemize}
\tightlist
\item
  R version \textgreater= 4.0.0
\item
  Operating System: Windows, macOS, or Linux
\item
  Memory: Minimum 4GB RAM recommended for large datasets
\end{itemize}

\newpage

\section{Git and GitHub Setup}\label{git-and-github-setup}

\subsection{Git Repository
Initialization}\label{git-repository-initialization}

The local Git repository was initialized with standard R package
structure:

\begin{Shaded}
\begin{Highlighting}[]
\FunctionTok{git}\NormalTok{ init}
\FunctionTok{git}\NormalTok{ add .}
\FunctionTok{git}\NormalTok{ commit }\AttributeTok{{-}m} \StringTok{"Initial commit: PathogenicityRanking R package"}
\end{Highlighting}
\end{Shaded}

\subsection{GitHub Repository
Configuration}\label{github-repository-configuration}

\subsubsection{Repository Details}\label{repository-details}

\begin{itemize}
\tightlist
\item
  \textbf{Account}: MohammadDeen
\item
  \textbf{Repository Name}: PathogenicityRanking
\item
  \textbf{Visibility}: Public
\item
  \textbf{URL}:
  \url{https://github.com/MohammadDeen/PathogenicityRanking}
\item
  \textbf{Description}: ``R package for composite pathogenicity scoring
  of missense variants''
\end{itemize}

\subsubsection{Branch Structure}\label{branch-structure}

\begin{itemize}
\tightlist
\item
  \textbf{Main Branch}: \texttt{main}
\item
  \textbf{Default Branch}: \texttt{main}
\item
  \textbf{Protection Rules}: None (open for contributions)
\end{itemize}

\subsection{Authentication Setup}\label{authentication-setup}

\subsubsection{SSH Key Configuration}\label{ssh-key-configuration}

A dedicated SSH key was generated for secure authentication:

\begin{Shaded}
\begin{Highlighting}[]
\CommentTok{\# Key generation}
\FunctionTok{ssh{-}keygen} \AttributeTok{{-}t}\NormalTok{ ed25519 }\AttributeTok{{-}C} \StringTok{"mohammaddeenhayatu@gmail.com"}
\CommentTok{\# Key name: pathorankinssh}

\CommentTok{\# SSH agent configuration}
\BuiltInTok{eval} \StringTok{"}\VariableTok{$(}\FunctionTok{ssh{-}agent} \AttributeTok{{-}s}\VariableTok{)}\StringTok{"}
\FunctionTok{ssh{-}add}\NormalTok{ pathorankinssh}
\end{Highlighting}
\end{Shaded}

\textbf{Public Key} (added to GitHub account):

\begin{verbatim}
ssh-ed25519 AAAAC3NzaC1lZDI1NTE5AAAAIFi03GKv01TM0c4q391YoDdOgFffeYlIKP2eZj/KUKFz mohammaddeenhayatu@gmail.com
\end{verbatim}

\subsubsection{Remote URL Configuration}\label{remote-url-configuration}

\begin{Shaded}
\begin{Highlighting}[]
\CommentTok{\# SSH{-}based remote URL}
\FunctionTok{git}\NormalTok{ remote set{-}url origin git@github.com:MohammadDeen/PathogenicityRanking.git}
\end{Highlighting}
\end{Shaded}

\subsubsection{Authentication
Verification}\label{authentication-verification}

\begin{Shaded}
\begin{Highlighting}[]
\FunctionTok{ssh} \AttributeTok{{-}T}\NormalTok{ git@github.com}
\CommentTok{\# Successful result: "Hi MohammadDeen! You\textquotesingle{}ve successfully authenticated"}
\end{Highlighting}
\end{Shaded}

\newpage

\section{Documentation}\label{documentation}

\subsection{README.md Features}\label{readme.md-features}

The README.md file provides comprehensive package documentation
including:

\subsubsection{Content Sections}\label{content-sections}

\begin{enumerate}
\def\labelenumi{\arabic{enumi}.}
\tightlist
\item
  \textbf{Package Title and Badges}

  \begin{itemize}
  \tightlist
  \item
    GitHub Actions CI/CD status badge
  \item
    MIT License badge
  \item
    Professional formatting
  \end{itemize}
\item
  \textbf{Installation Instructions}

  \begin{itemize}
  \tightlist
  \item
    devtools installation via GitHub
  \item
    Dependency management
  \item
    Quick start example
  \end{itemize}
\item
  \textbf{Function Documentation}

  \begin{itemize}
  \tightlist
  \item
    Parameter descriptions
  \item
    Required input format
  \item
    Output specifications
  \end{itemize}
\item
  \textbf{Usage Examples}

  \begin{itemize}
  \tightlist
  \item
    Basic function call
  \item
    Parameter configuration
  \item
    Expected outputs
  \end{itemize}
\end{enumerate}

\subsubsection{GitHub Badges}\label{github-badges}

\begin{Shaded}
\begin{Highlighting}[]
\CommentTok{[}\AlertTok{![R{-}CMD{-}check](https://github.com/MohammadDeen/PathogenicityRanking/actions/workflows/R{-}CMD{-}check.yaml/badge.svg)}\CommentTok{](https://github.com/MohammadDeen/PathogenicityRanking/actions/workflows/R{-}CMD{-}check.yaml)}
\CommentTok{[}\AlertTok{![License: MIT](https://img.shields.io/badge/License{-}MIT{-}yellow.svg)}\CommentTok{](https://opensource.org/licenses/MIT)}
\end{Highlighting}
\end{Shaded}

\subsection{Vignette Documentation}\label{vignette-documentation}

\subsubsection{File Location}\label{file-location}

\texttt{vignettes/PathogenicityRankingTutorial.Rmd}

\subsubsection{Content Structure}\label{content-structure}

\begin{itemize}
\tightlist
\item
  Package overview and purpose
\item
  Installation and loading instructions
\item
  Detailed usage examples
\item
  Best practices and recommendations
\end{itemize}

\subsubsection{Output Format}\label{output-format}

\begin{itemize}
\tightlist
\item
  HTML vignette for web viewing
\item
  Integration with R package documentation system
\end{itemize}

\subsection{License Documentation}\label{license-documentation}

\subsubsection{License Type}\label{license-type}

MIT License - permissive open-source license

\subsubsection{License Terms}\label{license-terms}

\begin{itemize}
\tightlist
\item
  Commercial use permitted
\item
  Modification permitted
\item
  Distribution permitted
\item
  Private use permitted
\item
  Liability limitations
\item
  Warranty disclaimers
\end{itemize}

\newpage

\section{Continuous Integration}\label{continuous-integration}

\subsection{GitHub Actions Workflow}\label{github-actions-workflow}

\subsubsection{Configuration File}\label{configuration-file}

\texttt{.github/workflows/R-CMD-check.yaml}

\subsubsection{Workflow Triggers}\label{workflow-triggers}

\begin{itemize}
\tightlist
\item
  Push events to main branch
\item
  Pull request events to main branch
\item
  Manual workflow dispatch
\end{itemize}

\subsubsection{Build Matrix}\label{build-matrix}

\begin{itemize}
\tightlist
\item
  \textbf{Platform}: Ubuntu Latest
\item
  \textbf{R Version}: Release
\item
  \textbf{Package Check}: R CMD check with CRAN standards
\end{itemize}

\subsubsection{Workflow Steps}\label{workflow-steps}

\begin{enumerate}
\def\labelenumi{\arabic{enumi}.}
\tightlist
\item
  \textbf{Checkout Code}: Uses
  \href{mailto:actions/checkout@v3}{\nolinkurl{actions/checkout@v3}}
\item
  \textbf{Setup R Environment}: Uses
  \href{mailto:r-lib/actions/setup-r@v2}{\nolinkurl{r-lib/actions/setup-r@v2}}
\item
  \textbf{Install System Dependencies}: Ubuntu package installation
\item
  \textbf{Install R Dependencies}: Automatic dependency resolution
\item
  \textbf{Package Build}: R CMD build execution
\item
  \textbf{Package Check}: R CMD check with --no-manual --as-cran flags
\end{enumerate}

\subsubsection{Benefits}\label{benefits}

\begin{itemize}
\tightlist
\item
  Automated testing on every code change
\item
  Early detection of package issues
\item
  CRAN compliance verification
\item
  Documentation generation validation
\end{itemize}

\newpage

\section{Security Implementation}\label{security-implementation}

\subsection{Git Security Measures}\label{git-security-measures}

\subsubsection{.gitignore Configuration}\label{gitignore-configuration}

The repository includes comprehensive .gitignore rules protecting:

\paragraph{R-Specific Files}\label{r-specific-files}

\begin{itemize}
\tightlist
\item
  History files: \texttt{.Rhistory}, \texttt{.Rapp.history}
\item
  Session data: \texttt{.RData}, \texttt{.RDataTmp}
\item
  User data: \texttt{.Ruserdata}
\end{itemize}

\paragraph{Development Files}\label{development-files}

\begin{itemize}
\tightlist
\item
  RStudio files: \texttt{.Rproj.user/}, \texttt{*.Rproj}
\item
  Build artifacts: \texttt{*.tar.gz}, \texttt{/*.Rcheck/}
\item
  Example files: \texttt{*-Ex.R}
\end{itemize}

\paragraph{Generated Content}\label{generated-content}

\begin{itemize}
\tightlist
\item
  Vignette outputs: \texttt{vignettes/*.html}, \texttt{vignettes/*.pdf}
\item
  Documentation: \texttt{docs/}
\item
  Package outputs: \texttt{*.pdf}, \texttt{*.png},
  \texttt{composite\_score\_results.csv}
\end{itemize}

\paragraph{System Files}\label{system-files}

\begin{itemize}
\tightlist
\item
  macOS: \texttt{.DS\_Store}
\item
  Windows: \texttt{Thumbs.db}
\item
  IDE: \texttt{.vscode/}, \texttt{.idea/}
\end{itemize}

\subsubsection{Authentication Security}\label{authentication-security}

\paragraph{SSH Key Implementation}\label{ssh-key-implementation}

\begin{itemize}
\tightlist
\item
  \textbf{Algorithm}: Ed25519 (modern, secure)
\item
  \textbf{Key Length}: 256-bit
\item
  \textbf{Passphrase}: Protected with user-defined passphrase
\item
  \textbf{Scope}: Repository-specific access
\end{itemize}

\paragraph{Token Security}\label{token-security}

\begin{itemize}
\tightlist
\item
  Personal Access Tokens removed from repository history
\item
  No embedded credentials in code or configuration
\item
  Secure authentication flow implemented
\end{itemize}

\subsection{Data Security
Considerations}\label{data-security-considerations}

\subsubsection{Input Data Protection}\label{input-data-protection}

\begin{itemize}
\tightlist
\item
  No sample data included in public repository
\item
  User responsible for data privacy compliance
\item
  Recommendation for data anonymization
\end{itemize}

\subsubsection{Output Security}\label{output-security}

\begin{itemize}
\tightlist
\item
  Generated files excluded from version control
\item
  Local file system protection responsibility
\item
  No automatic cloud uploads
\end{itemize}

\newpage

\section{Package Functionality}\label{package-functionality}

\subsection{Core Algorithm
Implementation}\label{core-algorithm-implementation}

\subsubsection{Data Processing Pipeline}\label{data-processing-pipeline}

\begin{enumerate}
\def\labelenumi{\arabic{enumi}.}
\item
  \textbf{File Format Detection}

\begin{Shaded}
\begin{Highlighting}[]
\ControlFlowTok{if}\NormalTok{ (}\FunctionTok{grepl}\NormalTok{(}\StringTok{"}\SpecialCharTok{\textbackslash{}\textbackslash{}}\StringTok{.(csv)$"}\NormalTok{, input\_file, }\AttributeTok{ignore.case =} \ConstantTok{TRUE}\NormalTok{)) \{}
\NormalTok{  df }\OtherTok{\textless{}{-}}\NormalTok{ readr}\SpecialCharTok{::}\FunctionTok{read\_csv}\NormalTok{(input\_file)}
\NormalTok{\} }\ControlFlowTok{else} \ControlFlowTok{if}\NormalTok{ (}\FunctionTok{grepl}\NormalTok{(}\StringTok{"}\SpecialCharTok{\textbackslash{}\textbackslash{}}\StringTok{.(txt)$"}\NormalTok{, input\_file, }\AttributeTok{ignore.case =} \ConstantTok{TRUE}\NormalTok{)) \{}
\NormalTok{  df }\OtherTok{\textless{}{-}}\NormalTok{ readr}\SpecialCharTok{::}\FunctionTok{read\_delim}\NormalTok{(input\_file, }\AttributeTok{delim =} \StringTok{"}\SpecialCharTok{\textbackslash{}t}\StringTok{"}\NormalTok{)}
\NormalTok{\} }\ControlFlowTok{else} \ControlFlowTok{if}\NormalTok{ (}\FunctionTok{grepl}\NormalTok{(}\StringTok{"}\SpecialCharTok{\textbackslash{}\textbackslash{}}\StringTok{.(xlsx)$"}\NormalTok{, input\_file, }\AttributeTok{ignore.case =} \ConstantTok{TRUE}\NormalTok{)) \{}
\NormalTok{  df }\OtherTok{\textless{}{-}}\NormalTok{ readxl}\SpecialCharTok{::}\FunctionTok{read\_excel}\NormalTok{(input\_file)}
\NormalTok{\}}
\end{Highlighting}
\end{Shaded}
\item
  \textbf{Score Normalization Function}

\begin{Shaded}
\begin{Highlighting}[]
\NormalTok{normalize }\OtherTok{\textless{}{-}} \ControlFlowTok{function}\NormalTok{(x) \{}
  \ControlFlowTok{if}\NormalTok{ (}\FunctionTok{all}\NormalTok{(}\FunctionTok{is.na}\NormalTok{(x)) }\SpecialCharTok{||} \FunctionTok{max}\NormalTok{(x, }\AttributeTok{na.rm =} \ConstantTok{TRUE}\NormalTok{) }\SpecialCharTok{==} \DecValTok{0}\NormalTok{) }\FunctionTok{return}\NormalTok{(x)}
\NormalTok{  x }\SpecialCharTok{/} \FunctionTok{max}\NormalTok{(x, }\AttributeTok{na.rm =} \ConstantTok{TRUE}\NormalTok{)}
\NormalTok{\}}
\end{Highlighting}
\end{Shaded}
\item
  \textbf{Composite Score Calculation}

\begin{Shaded}
\begin{Highlighting}[]
\NormalTok{df\_norm }\OtherTok{\textless{}{-}}\NormalTok{ df\_selected }\SpecialCharTok{\%\textgreater{}\%}
\NormalTok{  dplyr}\SpecialCharTok{::}\FunctionTok{mutate}\NormalTok{(}\FunctionTok{across}\NormalTok{(}\SpecialCharTok{{-}}\NormalTok{Variant, normalize)) }\SpecialCharTok{\%\textgreater{}\%}
\NormalTok{  dplyr}\SpecialCharTok{::}\FunctionTok{rowwise}\NormalTok{() }\SpecialCharTok{\%\textgreater{}\%}
\NormalTok{  dplyr}\SpecialCharTok{::}\FunctionTok{mutate}\NormalTok{(}\AttributeTok{CompositeScore =} \FunctionTok{mean}\NormalTok{(}\FunctionTok{c\_across}\NormalTok{(}\SpecialCharTok{{-}}\NormalTok{Variant), }\AttributeTok{na.rm =} \ConstantTok{TRUE}\NormalTok{))}
\end{Highlighting}
\end{Shaded}
\end{enumerate}

\subsubsection{Visualization
Implementation}\label{visualization-implementation}

\paragraph{Plot Specifications}\label{plot-specifications}

\begin{itemize}
\tightlist
\item
  \textbf{Chart Type}: Horizontal bar chart
\item
  \textbf{Color Scheme}: Professional blue (\#0072B2)
\item
  \textbf{Theme}: Minimal with custom formatting
\item
  \textbf{Sorting}: Descending by composite score
\item
  \textbf{Labels}: Clear axis titles and plot title
\end{itemize}

\paragraph{Export Specifications}\label{export-specifications}

\begin{itemize}
\tightlist
\item
  \textbf{PDF Output}: 10×6 inches, vector graphics
\item
  \textbf{PNG Output}: 10×6 inches, 300 DPI
\item
  \textbf{Format}: High-resolution for publication
\end{itemize}

\subsubsection{Error Handling}\label{error-handling}

\paragraph{Input Validation}\label{input-validation}

\begin{itemize}
\tightlist
\item
  File existence checking
\item
  Format support verification
\item
  Required column validation
\item
  Data type verification
\end{itemize}

\paragraph{Data Quality Checks}\label{data-quality-checks}

\begin{itemize}
\tightlist
\item
  Missing value handling
\item
  Zero-division protection
\item
  Normalization boundary conditions
\item
  Empty dataset handling
\end{itemize}

\newpage

\section{Installation and Usage}\label{installation-and-usage}

\subsection{Installation Process}\label{installation-process}

\subsubsection{Prerequisites}\label{prerequisites}

\begin{Shaded}
\begin{Highlighting}[]
\CommentTok{\# Ensure R version compatibility}
\NormalTok{R.version}\SpecialCharTok{$}\NormalTok{major }\SpecialCharTok{\textgreater{}=} \DecValTok{4}

\CommentTok{\# Install devtools if not available}
\ControlFlowTok{if}\NormalTok{ (}\SpecialCharTok{!}\FunctionTok{requireNamespace}\NormalTok{(}\StringTok{"devtools"}\NormalTok{, }\AttributeTok{quietly =} \ConstantTok{TRUE}\NormalTok{)) \{}
  \FunctionTok{install.packages}\NormalTok{(}\StringTok{"devtools"}\NormalTok{)}
\NormalTok{\}}
\end{Highlighting}
\end{Shaded}

\subsubsection{Package Installation}\label{package-installation}

\begin{Shaded}
\begin{Highlighting}[]
\CommentTok{\# Install PathogenicityRanking from GitHub}
\NormalTok{devtools}\SpecialCharTok{::}\FunctionTok{install\_github}\NormalTok{(}\StringTok{"MohammadDeen/PathogenicityRanking"}\NormalTok{)}

\CommentTok{\# Load the package}
\FunctionTok{library}\NormalTok{(PathogenicityRanking)}
\end{Highlighting}
\end{Shaded}

\subsubsection{Dependency Management}\label{dependency-management}

All required dependencies are automatically installed during package
installation.

\subsection{Usage Examples}\label{usage-examples}

\subsubsection{Basic Usage}\label{basic-usage}

\begin{Shaded}
\begin{Highlighting}[]
\CommentTok{\# Load the package}
\FunctionTok{library}\NormalTok{(PathogenicityRanking)}

\CommentTok{\# Run analysis with Excel input}
\NormalTok{results }\OtherTok{\textless{}{-}} \FunctionTok{run\_pathogenicity\_analysis}\NormalTok{(}
  \AttributeTok{input\_file =} \StringTok{"variants.xlsx"}\NormalTok{,}
  \AttributeTok{pdf\_output =} \StringTok{"pathogenicity\_ranking.pdf"}\NormalTok{,}
  \AttributeTok{png\_output =} \StringTok{"pathogenicity\_ranking.png"}\NormalTok{,}
  \AttributeTok{show\_plot =} \ConstantTok{TRUE}
\NormalTok{)}
\end{Highlighting}
\end{Shaded}

\subsubsection{Advanced Usage}\label{advanced-usage}

\begin{Shaded}
\begin{Highlighting}[]
\CommentTok{\# CSV input with custom output names}
\NormalTok{results }\OtherTok{\textless{}{-}} \FunctionTok{run\_pathogenicity\_analysis}\NormalTok{(}
  \AttributeTok{input\_file =} \StringTok{"my\_variants.csv"}\NormalTok{,}
  \AttributeTok{pdf\_output =} \StringTok{"custom\_ranking\_plot.pdf"}\NormalTok{,}
  \AttributeTok{png\_output =} \StringTok{"custom\_ranking\_plot.png"}\NormalTok{,}
  \AttributeTok{show\_plot =} \ConstantTok{FALSE}  \CommentTok{\# Don\textquotesingle{}t display in RStudio}
\NormalTok{)}

\CommentTok{\# Access results}
\FunctionTok{head}\NormalTok{(results)}
\FunctionTok{summary}\NormalTok{(results}\SpecialCharTok{$}\NormalTok{CompositeScore)}
\end{Highlighting}
\end{Shaded}

\subsubsection{Result Interpretation}\label{result-interpretation}

\begin{Shaded}
\begin{Highlighting}[]
\CommentTok{\# View top{-}ranked variants}
\NormalTok{top\_variants }\OtherTok{\textless{}{-}} \FunctionTok{head}\NormalTok{(results, }\DecValTok{10}\NormalTok{)}
\FunctionTok{print}\NormalTok{(top\_variants[, }\FunctionTok{c}\NormalTok{(}\StringTok{"Variant"}\NormalTok{, }\StringTok{"CompositeScore"}\NormalTok{)])}

\CommentTok{\# Export specific results}
\NormalTok{high\_risk }\OtherTok{\textless{}{-}}\NormalTok{ results[results}\SpecialCharTok{$}\NormalTok{CompositeScore }\SpecialCharTok{\textgreater{}} \FloatTok{0.8}\NormalTok{, ]}
\FunctionTok{write.csv}\NormalTok{(high\_risk, }\StringTok{"high\_risk\_variants.csv"}\NormalTok{, }\AttributeTok{row.names =} \ConstantTok{FALSE}\NormalTok{)}
\end{Highlighting}
\end{Shaded}

\newpage

\section{Future Development}\label{future-development}

\subsection{Recommended Enhancements}\label{recommended-enhancements}

\subsubsection{1. Sample Data
Integration}\label{sample-data-integration}

\begin{itemize}
\tightlist
\item
  Add example datasets in \texttt{data/} directory
\item
  Include various file formats for testing
\item
  Provide realistic pathogenicity score ranges
\item
  Create reproducible examples
\end{itemize}

\subsubsection{2. Unit Testing
Implementation}\label{unit-testing-implementation}

\begin{itemize}
\tightlist
\item
  Develop comprehensive test suite in \texttt{tests/testthat/}
\item
  Test edge cases and error conditions
\item
  Validate mathematical operations
\item
  Ensure cross-platform compatibility
\end{itemize}

\subsubsection{3. Documentation
Expansion}\label{documentation-expansion}

\begin{itemize}
\tightlist
\item
  Enhance vignettes with real-world case studies
\item
  Add mathematical documentation for scoring algorithms
\item
  Create video tutorials and screencasts
\item
  Develop API documentation with roxygen2
\end{itemize}

\subsubsection{4. Algorithm Improvements}\label{algorithm-improvements}

\begin{itemize}
\tightlist
\item
  Implement weighted composite scoring
\item
  Add additional pathogenicity predictors
\item
  Develop machine learning enhancements
\item
  Create customizable scoring schemes
\end{itemize}

\subsubsection{5. Visualization
Enhancements}\label{visualization-enhancements}

\begin{itemize}
\tightlist
\item
  Interactive plots with plotly
\item
  Multiple visualization options
\item
  Customizable color schemes and themes
\item
  Export format flexibility
\end{itemize}

\subsubsection{6. Performance
Optimization}\label{performance-optimization}

\begin{itemize}
\tightlist
\item
  Large dataset handling improvements
\item
  Memory usage optimization
\item
  Parallel processing implementation
\item
  Caching mechanisms
\end{itemize}

\subsection{CRAN Submission
Preparation}\label{cran-submission-preparation}

\subsubsection{Required Components}\label{required-components}

\begin{itemize}
\tightlist
\item[$\square$]
  Comprehensive unit tests (\textgreater80\% coverage)
\item[$\square$]
  CRAN policy compliance check
\item[$\square$]
  Cross-platform testing (Windows, macOS, Linux)
\item[$\square$]
  Documentation completeness verification
\item[$\square$]
  Example data inclusion
\item[$\square$]
  Performance benchmarking
\end{itemize}

\subsubsection{Submission Checklist}\label{submission-checklist}

\begin{itemize}
\tightlist
\item[$\square$]
  R CMD check passes with no errors, warnings, or notes
\item[$\square$]
  NEWS.md file with version history
\item[$\square$]
  CRAN comments document
\item[$\square$]
  Maintainer email verification
\item[$\square$]
  License compatibility confirmation
\end{itemize}

\newpage

\section{Troubleshooting}\label{troubleshooting}

\subsection{Common Issues and
Solutions}\label{common-issues-and-solutions}

\subsubsection{Installation Problems}\label{installation-problems}

\textbf{Issue}: Package dependencies fail to install\\
\textbf{Solution}: Update R version and install dependencies manually:

\begin{Shaded}
\begin{Highlighting}[]
\FunctionTok{install.packages}\NormalTok{(}\FunctionTok{c}\NormalTok{(}\StringTok{"dplyr"}\NormalTok{, }\StringTok{"ggplot2"}\NormalTok{, }\StringTok{"readr"}\NormalTok{, }\StringTok{"readxl"}\NormalTok{, }\StringTok{"fs"}\NormalTok{))}
\end{Highlighting}
\end{Shaded}

\textbf{Issue}: GitHub installation fails\\
\textbf{Solution}: Check internet connection and GitHub access:

\begin{Shaded}
\begin{Highlighting}[]
\CommentTok{\# Test GitHub connectivity}
\NormalTok{devtools}\SpecialCharTok{::}\FunctionTok{session\_info}\NormalTok{()}
\end{Highlighting}
\end{Shaded}

\subsubsection{Data Input Problems}\label{data-input-problems}

\textbf{Issue}: ``Required column not found'' error\\
\textbf{Solution}: Verify column names match exactly:

\begin{Shaded}
\begin{Highlighting}[]
\CommentTok{\# Check your column names}
\FunctionTok{colnames}\NormalTok{(your\_data)}
\CommentTok{\# Required: AAChange.refGeneWithVer, AlphaMissense\_score, etc.}
\end{Highlighting}
\end{Shaded}

\textbf{Issue}: File format not supported\\
\textbf{Solution}: Convert to supported format or check file extension

\subsubsection{Output Generation Issues}\label{output-generation-issues}

\textbf{Issue}: Plots not generated\\
\textbf{Solution}: Check file permissions and directory existence:

\begin{Shaded}
\begin{Highlighting}[]
\CommentTok{\# Verify write permissions}
\FunctionTok{file.access}\NormalTok{(}\StringTok{"."}\NormalTok{, }\DecValTok{2}\NormalTok{) }\SpecialCharTok{==} \DecValTok{0}  \CommentTok{\# Should return TRUE}
\end{Highlighting}
\end{Shaded}

\subsection{Performance
Considerations}\label{performance-considerations}

\subsubsection{Large Dataset Handling}\label{large-dataset-handling}

\begin{itemize}
\tightlist
\item
  For datasets \textgreater10,000 variants, consider chunked processing
\item
  Monitor memory usage with \texttt{pryr::mem\_used()}
\item
  Use data.table for extremely large files
\end{itemize}

\subsubsection{Computational
Requirements}\label{computational-requirements}

\begin{itemize}
\tightlist
\item
  Minimum 4GB RAM for typical datasets
\item
  SSD storage recommended for large file I/O
\item
  Multi-core systems benefit from parallel processing
\end{itemize}

\newpage

\section{Contact and Support}\label{contact-and-support}

\subsection{Author Information}\label{author-information}

\textbf{Mohammad Deen Hayatu}\\
Email:
\href{mailto:mohammaddeenhayatu@gmail.com}{\nolinkurl{mohammaddeenhayatu@gmail.com}}\\
GitHub: \href{https://github.com/MohammadDeen}{@MohammadDeen}

\subsection{Repository Information}\label{repository-information}

\textbf{GitHub Repository}:
\url{https://github.com/MohammadDeen/PathogenicityRanking}\\
\textbf{Issues and Bug Reports}:
\url{https://github.com/MohammadDeen/PathogenicityRanking/issues}\\
\textbf{Feature Requests}:
\url{https://github.com/MohammadDeen/PathogenicityRanking/discussions}

\subsection{Contributing}\label{contributing}

\subsubsection{Bug Reports}\label{bug-reports}

Please include: - R version and operating system - Package version -
Minimal reproducible example - Error messages and stack traces

\subsubsection{Feature Requests}\label{feature-requests}

Please provide: - Clear description of proposed feature - Use case
justification - Implementation suggestions - Compatibility
considerations

\subsubsection{Code Contributions}\label{code-contributions}

\begin{enumerate}
\def\labelenumi{\arabic{enumi}.}
\tightlist
\item
  Fork the repository
\item
  Create feature branch
\item
  Implement changes with tests
\item
  Submit pull request with description
\end{enumerate}

\subsection{Citation}\label{citation}

If you use this package in your research, please cite:

\begin{verbatim}
Hayatu, M.D. (2025). PathogenicityRanking: R package for composite 
pathogenicity scoring of missense variants. GitHub repository: 
https://github.com/MohammadDeen/PathogenicityRanking
\end{verbatim}

\begin{center}\rule{0.5\linewidth}{0.5pt}\end{center}

\textbf{Document Version}: 1.0\\
\textbf{Last Updated}: June 26, 2025\\
\textbf{Document Status}: Complete Setup Documentation

This documentation represents the comprehensive setup, implementation,
and usage guide for the PathogenicityRanking R package repository.

\end{document}
